% Created 2015-06-30 Tue 20:27
\documentclass[11pt]{article}
\usepackage[utf8]{inputenc}
\usepackage[T1]{fontenc}
\usepackage{fixltx2e}
\usepackage{graphicx}
\usepackage{longtable}
\usepackage{float}
\usepackage{wrapfig}
\usepackage{rotating}
\usepackage[normalem]{ulem}
\usepackage{amsmath}
\usepackage{textcomp}
\usepackage{marvosym}
\usepackage{wasysym}
\usepackage{amssymb}
\usepackage{hyperref}
\tolerance=1000
\usepackage{color}
\usepackage{listings}
\usepackage{minted}
\usepackage{xeCJK}
\setCJKmainfont{SHS UI PRChinaGB}
\setmainfont{Helvetica}
\usepackage[a3paper, margin=2cm]{geometry}
\setcounter{secnumdepth}{3}

\makeatletter
\def\verbatim{\tiny\@verbatim \frenchspacing\@vobeyspaces \@xverbatim}
\makeatother

\author{陈一鸣 解子傲}
\date{\today}
\title{MYSQL 查询优化实验报告}
\hypersetup{
  pdfkeywords={},
  pdfsubject={},
  pdfcreator={Emacs 24.5.1 (Org mode 8.2.10)}}
\begin{document}

\maketitle
\tableofcontents

\section{概述}
\label{sec-1}
\subsection{主要目的}
\label{sec-1-1}
通过对不同情况下查询语句的执行分析,巩固和加深对查询和查询优化相关理论知识的理解,
提高优化数据库系统的实践能力。
\subsection{优化手段}
\label{sec-1-2}
主要用到的优化手段与工具如下,参考资料也一并列出
\subsubsection{Explain}
\label{sec-1-2-1}
\begin{itemize}
\item \href{http://cbb.sjtu.edu.cn/course/database/lab6.htm}{实验6:数据库查询优化与范式}
\item \href{http://www.cnblogs.com/magialmoon/p/3439042.html}{MySQL优化—工欲善其事,必先利其器之EXPLAIN}
\end{itemize}
\subsubsection{建立索引}
\label{sec-1-2-2}
\begin{itemize}
\item 课本内容
\end{itemize}
\subsubsection{优化语句}
\label{sec-1-2-3}
\begin{itemize}
\item 课本内容
\end{itemize}
\subsubsection{慢查询日志}
\label{sec-1-2-4}
\begin{itemize}
\item \href{http://www.cnblogs.com/zhanjindong/p/3472804.html#manchaxunrizhi}{MySQL优化—工欲善其事,必先利其器(2)慢查询部分}
\end{itemize}
\subsubsection{Percona Toolkit}
\label{sec-1-2-5}
\begin{itemize}
\item \href{http://www.cnblogs.com/zhanjindong/p/3472804.html#PerconaToolkit}{MySQL优化—工欲善其事,必先利其器(2)Percona Toolkit 部分}
\item \href{https://www.percona.com/doc/percona-toolkit/2.1/}{PERCONA TOOLKIT DOCUMENTATION}
\end{itemize}
\subsubsection{Docker}
\label{sec-1-2-6}
\begin{itemize}
\item \href{https://www.docker.com}{Docker}
\item \href{https://registry.hub.docker.com/_/mysql/}{Docker MySQL repository}
\item \url{http://dockerpool.com/static/books/docker_practice/appendix_repo/mysql.html}
\item \url{http://blog.flux7.com/blogs/docker/docker-tutorial-series-part-1-an-introduction}
\end{itemize}
\subsection{优化思路}
\label{sec-1-3}
由于本次实验的数据量较小,各种查询(嵌套查询除外)几乎都能在 0.1 秒以下的时间运
行完毕,而这个时间很大程度还受到其他因素的影响(如:机器当前负载等)。而这么短的
时间也无法进行慢查询日志的记录, Percona Toolkit 可以做到精确到纳秒,但仍要基于
慢日志查询。故本次优化不以执行 SQL 语句的时间为标准,而是以 Explain 语句结果中的
rows 指标为标准, rows 越小,认为效率越高。主要的优化手段也只用到了 \textbf{Explain},
\textbf{Index} 和 \textbf{语句的优化} 。
\subsubsection{优化过程}
\label{sec-1-3-1}
\begin{enumerate}
\item Explain 原查询
\item 结合 Explain 结果进行优化
\item Explain 优化后的查询
\end{enumerate}
\section{实验环境}
\label{sec-2}
本次实验为了统一两人的环境,将实验环境搭建在了实验室内网的一台 PC 上,不仅如此,
还将 MySQL 搭建在了 Docker 虚拟机中。每次实验前用 Python 脚本将数据库重置一遍,
严格控制变量。(另:此处能用 DockerFile 进行初始化,但因初始化的同时要进行数据的
初始化,故选择 Python 脚本)

以下是实验环境的具体信息

\subsection{PC 信息}
\label{sec-2-1}
\begin{itemize}
\item System Information
\begin{itemize}
\item Manufacturer: Dell Inc.
\item Product Name: Inspiron 620
\end{itemize}
\item OS

Ubuntu 14.04.2 LTS

\item CPU

Intel(R) Core(TM) i5-2320 CPU @ 3.00GHz

\item Memory

4 GB 1333 MHz

\item Hard Drive

1 TB 7200 rpm
\end{itemize}
\subsection{Docker}
\label{sec-2-2}
\subsubsection{Docker 简介}
\label{sec-2-2-1}
Docker 是一个开源的应用容器引擎,让开发者可以打包他们的应用以及依赖包到一个可移
植的容器中,然后发布到任何流行的 Linux 机器上,也可以实现虚拟化。容器是完全使用
沙箱机制,相互之间不会有任何接口(类似 iPhone 的 app)。几乎没有性能开销,可以很
容易地在机器和数据中心中运行。最重要的是,他们不依赖于任何语言、框架包括系统。\footnote{\url{http://baike.baidu.com/view/11854949.htm}}
\subsubsection{Docker Mysql}
\label{sec-2-2-2}
以下是用 Docker 建立本次实验环境的语句(仅供参考):
\begin{itemize}
\item Pull
\lstset{breaklines,keywordstyle=\color{black}\bfseries,basicstyle=\ttfamily\scriptsize,language=sh,label= ,caption= ,numbers=none}
\begin{lstlisting}
docker pull dl.dockerpool.com:5000/mysql:latest
docker tag dl.dockerpool.com:5000/mysql:latest mysql:latest
\end{lstlisting}
\item Run
\lstset{breaklines,keywordstyle=\color{black}\bfseries,basicstyle=\ttfamily\scriptsize,language=sh,label= ,caption= ,numbers=none}
\begin{lstlisting}
docker run --name test-mysql -e MYSQL_ROOT_PASSWORD=123123 -p 3306:3306 -d mysql
\end{lstlisting}
\item Shell
\lstset{breaklines,keywordstyle=\color{black}\bfseries,basicstyle=\ttfamily\scriptsize,language=sh,label= ,caption= ,numbers=none}
\begin{lstlisting}
docker exec -it test-mysql bash
\end{lstlisting}
\end{itemize}
\subsubsection{Docker info}
\label{sec-2-2-3}
本次 Docker 的运行环境

\lstset{breaklines,keywordstyle=\color{black}\bfseries,basicstyle=\ttfamily\scriptsize,language=sh,label= ,caption= ,numbers=none}
\begin{lstlisting}
docker info
\end{lstlisting}

\begin{verbatim}
Storage Driver: aufs
 Root Dir: /var/lib/docker/aufs
 Backing Filesystem: extfs
 Dirs: 29
 Dirperm1 Supported: true
Execution Driver: native-0.2
Kernel Version: 3.16.0-38-generic
Operating System: Ubuntu 14.04.2 LTS
CPUs: 4
Total Memory: 3.844 GiB
\end{verbatim}
\subsubsection{使用 Docker 的好处}
\label{sec-2-2-4}
\begin{enumerate}
\item 多个 MySQL 实例并行

可以同时在一台机上运行多个 Docker 实例(即多个 MySQL 实例),两个人可同时进行
实验,且互不影响。

\item 方便重置

只用执行几段代码即可重新搭建出一个新的实验环境,结合 Python 脚本做到任意操作
数据库都能轻松回到初始状态。

\item 与本机隔离

由于跑在虚拟机中,故不必担心影响到服务器本身的环境。
\end{enumerate}
\subsection{Python 脚本}
\label{sec-2-3}
详见本项目 Github 主页\footnote{\url{https://github.com/dsdshcym/Database-Query-Optimization}}
\subsubsection{reset.py}
\label{sec-2-3-1}

调用 \texttt{init.sql} 中的 sql 语句将数据库重置。
\subsubsection{update.py}
\label{sec-2-3-2}

利用数据库中已有的表信息,从 csv 文件中提取出相应的数据载入到数据库中。(因此只
需要完成表的设计即能自动录入,即便表结构变化也不会受到影响)
\section{实验内容}
\label{sec-3}
\subsection{垂直分割}
\label{sec-3-1}
\subsubsection{数据含义分析}
\label{sec-3-1-1}
结合 csv 文件中的数据名和数据值得含义如下:
\begin{itemize}
\item 标签分析
\label{sec-3-1-1-1}
\begin{description}
\item[{shop\_id}] 编号
\item[{name}] 店名
\item[{alias}] 又名
\item[{province}] 省份
\item[{city}] 城市
\item[{city\_pinyin}] 城市拼音
\item[{city\_id}] 城市编号
\item[{area}] 城区
\item[{big\_cate}] 大类
\item[{big\_cate\_id}] 大类编号
\item[{small\_cate}] 小类
\item[{small\_cate\_id}] 小类编号
\item[{address}] 地址
\item[{business\_area}] 商区
\item[{phone}] 电话
\item[{hours}] 开业时间
\item[{avg\_price}] 平均价格
\item[{stars}] 星级
\item[{photos}] 照片
\item[{description}] 描述
\item[{tags}] 标签
\item[{map\_type}] 地图类型
\item[{original\_latitude}] 原始纬度
\item[{original\_longitude}] 原始经度
\item[{google\_longitude}] 谷歌经度
\item[{google\_latitude}] 谷歌纬度
\item[{navigation}] 导航
\item[{traffic}] 交通
\item[{atmosphere}] 气氛
\item[{characteristics}] 特色
\item[{payment}] 付款
\item[{product\_rating}] 产品评价
\item[{environment\_rating}] 环境评价
\item[{service\_rating}] 服务评价
\item[{all\_remarks}] 所有评价数
\item[{very\_good\_remarks}] 「非常好」评价数
\item[{good\_remarks}] 「好」评价数
\item[{common\_remarks}] 「一般」评价数
\item[{bad\_remarks}] 「差」评价数
\item[{very\_bad\_remarks}] 「非常差」评价数
\item[{recommended\_dishes}] 推荐菜品
\item[{nearby\_shops}] 附近商铺
\item[{is\_chains}] 是连锁店
\item[{group}] 团购
\item[{card}] 优惠券
\end{description}
\end{itemize}
\subsubsection{分割}
\label{sec-3-1-2}
分析后分为 14 个表,如下(主键用下划线标出)
\begin{itemize}
\item 基本信息 (basic)
\begin{description}
\item[{\uline{shop\_id}}] 编号
\item[{name}] 店名
\item[{alias}] 又名
\item[{address}] 地址
\item[{phone}] 电话
\item[{hours}] 开业时间
\item[{avg\_price}] 平均价格
\item[{payment}] 付款
\item[{is\_chains}] 是连锁店
\end{description}
\item 城市相关
\begin{itemize}
\item 所在地 (shop\_id\_area)
\begin{description}
\item[{\uline{shop\_id}}] 编号
\item[{area}] 城区
\item[{business\_area}] 商区
\end{description}
\item 店-城市 (shop\_id\_city\_id)
\begin{description}
\item[{\uline{shop\_id}}] 编号
\item[{\uline{city\_id}}] 城市编号
\end{description}
\item 城市 (city\_id\_city)
\begin{description}
\item[{\uline{city\_id}}] 城市编号
\item[{city}] 城市
\end{description}
\item 城市、城市拼音 (city\_id\_city\_pinyin)
\begin{description}
\item[{\uline{city\_id}}] 城市编号
\item[{city\_pinyin}] 城市拼音
\end{description}
\item 城市、省份 (city\_id\_province)
\begin{description}
\item[{\uline{city\_id}}] 城市编号
\item[{province}] 省份
\end{description}
\end{itemize}
\item 分类
\begin{itemize}
\item 编号-小类 (shop\_id\_small\_cate\_id)
\begin{description}
\item[{\uline{shop\_id}}] 编号
\item[{small\_cate}] 小类
\end{description}
\item 小类 (small\_cate\_id\_small\_cate)
\begin{description}
\item[{\uline{small\_cate\_id}}] 小类编号
\item[{small\_cate}] 小类
\end{description}
\item 编号-大类 (shop\_id\_big\_cate\_id)
\begin{description}
\item[{\uline{shop\_id}}] 编号
\item[{small\_cate}] 大类
\end{description}
\item 大类 (big\_cate\_id\_big\_cate)
\begin{description}
\item[{\uline{big\_cate\_id}}] 大类编号
\item[{big\_cate}] 大类
\end{description}
\end{itemize}
\item 地图类型 (map\_info)
\begin{description}
\item[{\uline{shop\_id}}] 编号
\item[{map\_type}] 地图类型
\item[{original\_latitude}] 原始纬度
\item[{original\_longitude}] 原始经度
\item[{google\_longitude}] 谷歌经度
\item[{google\_latitude}] 谷歌纬度
\item[{traffic}] 交通
\end{description}
\item 大众相关 (dazhong)
\begin{description}
\item[{\uline{shop\_id}}] 编号
\item[{navigation}] 网站导航
\item[{recommended\_dishes}] 推荐菜品
\item[{characteristics}] 特色
\item[{stars}] 星级
\item[{photos}] 照片
\item[{description}] 描述
\item[{tags}] 标签
\item[{atmosphere}] 气氛
\item[{nearby\_shops}] 附近商铺
\end{description}
\item 评价 (remark)
\begin{description}
\item[{\uline{shop\_id}}] 编号
\item[{product\_rating    }] 产品评价
\item[{environment\_rating}] 环境评价
\item[{service\_rating    }] 服务评价
\item[{all\_remarks       }] 所有评价数
\item[{very\_good\_remarks }] 「非常好」评价数
\item[{good\_remarks      }] 「好」评价数
\item[{common\_remarks    }] 「一般」评价数
\item[{bad\_remarks       }] 「差」评价数
\item[{very\_bad\_remarks  }] 「非常差」评价数
\end{description}
\item 优惠 (discount)
\begin{description}
\item[{\uline{shop\_id}}] 编号
\item[{group}] 团购
\item[{card}] 优惠券
\end{description}
\end{itemize}
\subsubsection{建表语句}
\label{sec-3-1-3}
见本项目 Github 页面\footnote{\url{https://github.com/dsdshcym/Database-Query-Optimization/blob/master/init.sql}}

其中所有 Char, Float 变量均为定长,保证效率\footnote{\url{http://coolshell.cn/articles/1846.html} 第 15 条}
\subsection{查询优化}
\label{sec-3-2}
\subsubsection{单表操作}
\label{sec-3-2-1}
\begin{itemize}
\item 查询表中的所有字段
\label{sec-3-2-1-1}
\begin{enumerate}
\item 查询 basic 表
\begin{enumerate}
\item 原查询

实际应用中会出现查询所有店铺信息的情况,比如查询 basic
表中的所有字段的结果

\lstset{breaklines,keywordstyle=\color{black}\bfseries,basicstyle=\ttfamily\scriptsize,language=SQL,label= ,caption= ,numbers=none}
\begin{lstlisting}
explain select *
from basic;
\end{lstlisting}

\begin{verbatim}
+----+-------------+-------+------+---------------+------+---------+------+------+-------+
| id | select_type | table | type | possible_keys | key  | key_len | ref  | rows | Extra |
+----+-------------+-------+------+---------------+------+---------+------+------+-------+
| 1  | SIMPLE      | basic | ALL  | NULL          | NULL | NULL    | NULL | 963  | NULL  |
+----+-------------+-------+------+---------------+------+---------+------+------+-------+
1 rows in set (0.05 sec)
\end{verbatim}

可见 rows 为 963

\item 无法优化

查询所有字段需遍历表中所有元组,无法再进一步优化了
\end{enumerate}

\item 查询所有小类
\begin{enumerate}
\item 原查询

\lstset{breaklines,keywordstyle=\color{black}\bfseries,basicstyle=\ttfamily\scriptsize,language=SQL,label= ,caption= ,numbers=none}
\begin{lstlisting}
explain select *
from small_cate_id_small_cate;
\end{lstlisting}

\begin{verbatim}
+----+-------------+--------------------------+------+---------------+------+---------+------+------+-------+
| id | select_type | table                    | type | possible_keys | key  | key_len | ref  | rows | Extra |
+----+-------------+--------------------------+------+---------------+------+---------+------+------+-------+
|  1 | SIMPLE      | small_cate_id_small_cate | ALL  | NULL          | NULL | NULL    | NULL |   37 | NULL  |
+----+-------------+--------------------------+------+---------------+------+---------+------+------+-------+
1 row in set (0.03 sec)
\end{verbatim}

\item 无法优化

查询所有字段需遍历表中所有元组,无法再进一步优化了
\end{enumerate}
\end{enumerate}
\item 查询表中的指定字段
\label{sec-3-2-1-2}
\begin{enumerate}
\item 查询所有店铺的名称
\begin{enumerate}
\item 原查询

\lstset{breaklines,keywordstyle=\color{black}\bfseries,basicstyle=\ttfamily\scriptsize,language=SQL,label= ,caption= ,numbers=none}
\begin{lstlisting}
explain select name
from basic;
\end{lstlisting}

\begin{verbatim}
+----+-------------+-------+------+---------------+------+---------+------+------+-------+
| id | select_type | table | type | possible_keys | key  | key_len | ref  | rows | Extra |
+----+-------------+-------+------+---------------+------+---------+------+------+-------+
| 1  | SIMPLE      | basic | ALL  | NULL          | NULL | NULL    | NULL | 963  | NULL  |
+----+-------------+-------+------+---------------+------+---------+------+------+-------+
1 rows in set (0.05 sec)
\end{verbatim}

可见 rows 为 963

\item 无法优化

该查询需遍历表中所有元组,无法再进一步优化了
\end{enumerate}

\item 查询优惠表中的所有团购信息
\begin{enumerate}
\item 原查询

\lstset{breaklines,keywordstyle=\color{black}\bfseries,basicstyle=\ttfamily\scriptsize,language=SQL,label= ,caption= ,numbers=none}
\begin{lstlisting}
explain select group_info
from discount;
\end{lstlisting}

\begin{verbatim}
+----+-------------+----------+------+---------------+------+---------+------+------+-------+
| id | select_type | table    | type | possible_keys | key  | key_len | ref  | rows | Extra |
+----+-------------+----------+------+---------------+------+---------+------+------+-------+
|  1 | SIMPLE      | discount | ALL  | NULL          | NULL | NULL    | NULL | 1000 | NULL  |
+----+-------------+----------+------+---------------+------+---------+------+------+-------+
1 row in set (0.00 sec)
\end{verbatim}

\item 无法优化

该查询需遍历表中所有元组,无法再进一步优化了
\end{enumerate}
\end{enumerate}
\item 查询表中没有重复的字段(distinct)的使用
\label{sec-3-2-1-3}
\begin{enumerate}
\item 查询不重名的所有店铺名称
\begin{enumerate}
\item 原查询

\lstset{breaklines,keywordstyle=\color{black}\bfseries,basicstyle=\ttfamily\scriptsize,language=SQL,label= ,caption= ,numbers=none}
\begin{lstlisting}
explain select distinct name
from basic;
\end{lstlisting}

\begin{verbatim}
+----+-------------+-------+------+---------------+------+---------+------+------+-----------------+
| id | select_type | table | type | possible_keys | key  | key_len | ref  | rows | Extra           |
+----+-------------+-------+------+---------------+------+---------+------+------+-----------------+
| 1  | SIMPLE      | basic | ALL  | NULL          | NULL | NULL    | NULL | 963  | Using temporary |
+----+-------------+-------+------+---------------+------+---------+------+------+-----------------+
1 rows in set (0.05 sec)
\end{verbatim}

可见 rows 为 963

\item 无法优化

该查询需遍历表中所有元组,无法再进一步优化了
\end{enumerate}

\item 查询所有地图类型
\begin{enumerate}
\item 原查询

\lstset{breaklines,keywordstyle=\color{black}\bfseries,basicstyle=\ttfamily\scriptsize,language=SQL,label= ,caption= ,numbers=none}
\begin{lstlisting}
explain select distinct map_type
from map_info;
\end{lstlisting}

\begin{verbatim}
+----+-------------+----------+------+---------------+------+---------+------+------+-----------------+
| id | select_type | table    | type | possible_keys | key  | key_len | ref  | rows | Extra           |
+----+-------------+----------+------+---------------+------+---------+------+------+-----------------+
|  1 | SIMPLE      | map_info | ALL  | NULL          | NULL | NULL    | NULL | 1000 | Using temporary |
+----+-------------+----------+------+---------------+------+---------+------+------+-----------------+
1 row in set (0.00 sec)
\end{verbatim}

\item 添加索引

\lstset{breaklines,keywordstyle=\color{black}\bfseries,basicstyle=\ttfamily\scriptsize,language=SQL,label= ,caption= ,numbers=none}
\begin{lstlisting}
create index index_map_type on map_info(map_type);
\end{lstlisting}

再次查询

\begin{verbatim}
+----+-------------+----------+-------+----------------+----------------+---------+------+------+--------------------------+
| id | select_type | table    | type  | possible_keys  | key            | key_len | ref  | rows | Extra                    |
+----+-------------+----------+-------+----------------+----------------+---------+------+------+--------------------------+
|  1 | SIMPLE      | map_info | range | index_map_type | index_map_type | 1       | NULL |    5 | Using index for group-by |
+----+-------------+----------+-------+----------------+----------------+---------+------+------+--------------------------+
1 row in set (0.02 sec)
\end{verbatim}

rows 变为 5 ,效率显著提高
\end{enumerate}
\end{enumerate}
\item 条件查询各表主键的字段(单值查询或范围查询)
\label{sec-3-2-1-4}
\begin{enumerate}
\item 用一个精确的店编号去查找店铺信息
\begin{enumerate}
\item 原查询

比如查找 basic 中 shop\_id=10328540 的店铺信息

\lstset{breaklines,keywordstyle=\color{black}\bfseries,basicstyle=\ttfamily\scriptsize,language=SQL,label= ,caption= ,numbers=none}
\begin{lstlisting}
explain select *
from basic
where shop_id=10328540;
\end{lstlisting}

\begin{verbatim}
+----+-------------+-------+-------+---------------+---------+---------+-------+------+-------+
| id | select_type | table | type  | possible_keys | key     | key_len | ref   | rows | Extra |
+----+-------------+-------+-------+---------------+---------+---------+-------+------+-------+
| 1  | SIMPLE      | basic | const | PRIMARY       | PRIMARY | 4       | const | 1    | NULL  |
+----+-------------+-------+-------+---------------+---------+---------+-------+------+-------+
1 rows in set (0.05 sec)
\end{verbatim}

可见 rows 已经为 1

\item 无需优化

因为表创建时已经自带以主键为关键值的索引,无需优化
\end{enumerate}
\item 用店编号范围去查找一部分店铺信息
\begin{enumerate}
\item 原查询

比如查找 basic 中 shop\_id>10328540 and shop\_id<10329940 的店铺信息

\lstset{breaklines,keywordstyle=\color{black}\bfseries,basicstyle=\ttfamily\scriptsize,language=SQL,label= ,caption= ,numbers=none}
\begin{lstlisting}
explain select *
from basic
where shop_id>10328540 and shop_id<10329940;
\end{lstlisting}

\begin{verbatim}
+----+-------------+-------+-------+---------------+---------+---------+------+------+-------------+
| id | select_type | table | type  | possible_keys | key     | key_len | ref  | rows | Extra       |
+----+-------------+-------+-------+---------------+---------+---------+------+------+-------------+
| 1  | SIMPLE      | basic | range | PRIMARY       | PRIMARY | 4       | NULL | 36   | Using where |
+----+-------------+-------+-------+---------------+---------+---------+------+------+-------------+
1 rows in set (0.01 sec)
\end{verbatim}

可见 rows 已经为36

\item 无需优化

因为表创建时已经自带以主键为关键值的索引,无需优化
\end{enumerate}
\end{enumerate}
\item 条件查询各表中普通字段(单值查询或范围查询)
\label{sec-3-2-1-5}
\begin{enumerate}
\item 用店名来查找店铺信息
\begin{enumerate}
\item 原查询
比如查找 base 中 name 为「林师傅」的结果

\lstset{breaklines,keywordstyle=\color{black}\bfseries,basicstyle=\ttfamily\scriptsize,language=SQL,label= ,caption= ,numbers=none}
\begin{lstlisting}
explain select *
from basic
where name='林师傅';
\end{lstlisting}

\begin{verbatim}
+----+-------------+-------+------+---------------+------+---------+------+------+-------------+
| id | select_type | table | type | possible_keys | key  | key_len | ref  | rows | Extra       |
+----+-------------+-------+------+---------------+------+---------+------+------+-------------+
| 1  | SIMPLE      | basic | ALL  | NULL          | NULL | NULL    | NULL | 963  | Using where |
+----+-------------+-------+------+---------------+------+---------+------+------+-------------+
1 rows in set (0.03 sec)
\end{verbatim}

可见 rows 为 963

\item 对 name 进行索引

\lstset{breaklines,keywordstyle=\color{black}\bfseries,basicstyle=\ttfamily\scriptsize,language=SQL,label= ,caption= ,numbers=none}
\begin{lstlisting}
create index index_name on basic(name);
\end{lstlisting}

再次查找

\begin{verbatim}
+----+-------------+-------+------+---------------+------------+---------+-------+------+-----------------------+
| id | select_type | table | type | possible_keys | key        | key_len | ref   | rows | Extra                 |
+----+-------------+-------+------+---------------+------------+---------+-------+------+-----------------------+
| 1  | SIMPLE      | basic | ref  | index_name    | index_name | 150     | const | 1    | Using index condition |
+----+-------------+-------+------+---------------+------------+---------+-------+------+-----------------------+
1 rows in set (0.04 sec)
\end{verbatim}

可见 rows 已经变为 1,是优化前的0.1\%
\end{enumerate}
\item 查找人均消费在某一范围内的的店铺名称
\begin{enumerate}
\item 原查询

比如查找 base 中 avg\_price<50 的结果

\lstset{breaklines,keywordstyle=\color{black}\bfseries,basicstyle=\ttfamily\scriptsize,language=SQL,label= ,caption= ,numbers=none}
\begin{lstlisting}
explain select name
from basic
where avg_price<50;
\end{lstlisting}

\begin{verbatim}
+----+-------------+-------+------+---------------+------+---------+------+------+-------------+
| id | select_type | table | type | possible_keys | key  | key_len | ref  | rows | Extra       |
+----+-------------+-------+------+---------------+------+---------+------+------+-------------+
| 1  | SIMPLE      | basic | ALL  | NULL          | NULL | NULL    | NULL | 963  | Using where |
+----+-------------+-------+------+---------------+------+---------+------+------+-------------+
1 rows in set (0.03 sec)
\end{verbatim}

可见 rows 为 963

\item 对 avg\_price,name 进行索引

\lstset{breaklines,keywordstyle=\color{black}\bfseries,basicstyle=\ttfamily\scriptsize,language=SQL,label= ,caption= ,numbers=none}
\begin{lstlisting}
create index index_name_price on basic(avg_price,name);
\end{lstlisting}

再次查找

\begin{verbatim}
+----+-------------+-------+-------+------------------+------------------+---------+------+------+--------------------------+
| id | select_type | table | type  | possible_keys    | key              | key_len | ref  | rows | Extra                    |
+----+-------------+-------+-------+------------------+------------------+---------+------+------+--------------------------+
| 1  | SIMPLE      | basic | range | index_name_price | index_name_price | 2       | NULL | 617  | Using where; Using index |
+----+-------------+-------+-------+------------------+------------------+---------+------+------+--------------------------+
1 rows in set (0.05 sec)
\end{verbatim}

可见 rows 已经变为617,是优化前的64\%
\end{enumerate}
\end{enumerate}
\item 一个表中多个字段条件查询(单值查询或范围查询)
\label{sec-3-2-1-6}
\begin{enumerate}
\item 多条件查询 remark 表中的 shop\_id
\begin{enumerate}
\item 原查询

\lstset{breaklines,keywordstyle=\color{black}\bfseries,basicstyle=\ttfamily\scriptsize,language=SQL,label= ,caption= ,numbers=none}
\begin{lstlisting}
explain select shop_id
from remark
where product_rating > 8.5 and environment_rating > 8.5;
\end{lstlisting}

\begin{verbatim}
+----+-------------+--------+------+---------------+------+---------+------+------+-------------+
| id | select_type | table  | type | possible_keys | key  | key_len | ref  | rows | Extra       |
+----+-------------+--------+------+---------------+------+---------+------+------+-------------+
|  1 | SIMPLE      | remark | ALL  | NULL          | NULL | NULL    | NULL | 1000 | Using where |
+----+-------------+--------+------+---------------+------+---------+------+------+-------------+
1 row in set (0.00 sec)
\end{verbatim}

可见 rows 为 1000

\item 添加索引

\lstset{breaklines,keywordstyle=\color{black}\bfseries,basicstyle=\ttfamily\scriptsize,language=SQL,label= ,caption= ,numbers=none}
\begin{lstlisting}
create index index_product_rating on remark(product_rating);
create index index_environment_rating on remark(environment_rating);
\end{lstlisting}

\item 优化后

\lstset{breaklines,keywordstyle=\color{black}\bfseries,basicstyle=\ttfamily\scriptsize,language=SQL,label= ,caption= ,numbers=none}
\begin{lstlisting}
explain select shop_id
from remark
where product_rating > 8.5 and environment_rating > 8.5;
\end{lstlisting}

\begin{verbatim}
+----+-------------+--------+-------+-----------------------------------------------+----------------------+---------+------+------+------------------------------------+
| id | select_type | table  | type  | possible_keys                                 | key                  | key_len | ref  | rows | Extra                              |
+----+-------------+--------+-------+-----------------------------------------------+----------------------+---------+------+------+------------------------------------+
|  1 | SIMPLE      | remark | range | index_product_rating,index_environment_rating | index_product_rating | 5       | NULL |   52 | Using index condition; Using where |
+----+-------------+--------+-------+-----------------------------------------------+----------------------+---------+------+------+------------------------------------+
1 row in set (0.00 sec)
\end{verbatim}

优化后 rows 变为 52 ,效率显著提高
\end{enumerate}

\item 通过 name 和 alias 查询 phone
\begin{enumerate}
\item 原查询

\lstset{breaklines,keywordstyle=\color{black}\bfseries,basicstyle=\ttfamily\scriptsize,language=SQL,label= ,caption= ,numbers=none}
\begin{lstlisting}
explain select phone from basic
where name = '巫山烤全鱼' and alias = '重庆鸡公煲';
\end{lstlisting}

\begin{verbatim}
+----+-------------+-------+------+---------------+------+---------+------+------+-------------+
| id | select_type | table | type | possible_keys | key  | key_len | ref  | rows | Extra       |
+----+-------------+-------+------+---------------+------+---------+------+------+-------------+
|  1 | SIMPLE      | basic | ALL  | NULL          | NULL | NULL    | NULL |  963 | Using where |
+----+-------------+-------+------+---------------+------+---------+------+------+-------------+
1 row in set (0.00 sec)
\end{verbatim}

可见 rows 为 963 ,有优化空间

\item 添加索引

\lstset{breaklines,keywordstyle=\color{black}\bfseries,basicstyle=\ttfamily\scriptsize,language=SQL,label= ,caption= ,numbers=none}
\begin{lstlisting}
create index index_name on basic(name);
\end{lstlisting}

\begin{verbatim}
+----+-------------+-------+------+---------------+------------+---------+-------+------+------------------------------------+
| id | select_type | table | type | possible_keys | key        | key_len | ref   | rows | Extra                              |
+----+-------------+-------+------+---------------+------------+---------+-------+------+------------------------------------+
|  1 | SIMPLE      | basic | ref  | index_name    | index_name | 150     | const |    3 | Using index condition; Using where |
+----+-------------+-------+------+---------------+------------+---------+-------+------+------------------------------------+
1 row in set (0.00 sec)
\end{verbatim}

rows 降为 3

\lstset{breaklines,keywordstyle=\color{black}\bfseries,basicstyle=\ttfamily\scriptsize,language=SQL,label= ,caption= ,numbers=none}
\begin{lstlisting}
create index index_alias on basic(alias);
\end{lstlisting}

\begin{verbatim}
+----+-------------+-------+------+------------------------+-------------+---------+-------+------+------------------------------------+
| id | select_type | table | type | possible_keys          | key         | key_len | ref   | rows | Extra                              |
+----+-------------+-------+------+------------------------+-------------+---------+-------+------+------------------------------------+
|  1 | SIMPLE      | basic | ref  | index_name,index_alias | index_alias | 120     | const |    1 | Using index condition; Using where |
+----+-------------+-------+------+------------------------+-------------+---------+-------+------+------------------------------------+
1 row in set (0.07 sec)
\end{verbatim}

rows 降为 1 ,效率显著提高

\item 更改查询语句顺序 + 索引

由于 alias 上大多为空值,重复率较低,可先查询 alias 再查询 name

\lstset{breaklines,keywordstyle=\color{black}\bfseries,basicstyle=\ttfamily\scriptsize,language=SQL,label= ,caption= ,numbers=none}
\begin{lstlisting}
create index index_alias on basic(alias);

explain select phone from basic
where alias = '重庆鸡公煲' and name = '巫山烤全鱼';
\end{lstlisting}

\begin{verbatim}
+----+-------------+-------+------+---------------+-------------+---------+-------+------+------------------------------------+
| id | select_type | table | type | possible_keys | key         | key_len | ref   | rows | Extra                              |
+----+-------------+-------+------+---------------+-------------+---------+-------+------+------------------------------------+
|  1 | SIMPLE      | basic | ref  | index_alias   | index_alias | 120     | const |    1 | Using index condition; Using where |
+----+-------------+-------+------+---------------+-------------+---------+-------+------+------------------------------------+
1 row in set (0.00 sec)
\end{verbatim}

rows 降为 1 ,效率显著提高
\end{enumerate}

\item 用店名和人均消费值来查找店铺信息
\begin{enumerate}
\item 原查询

\lstset{breaklines,keywordstyle=\color{black}\bfseries,basicstyle=\ttfamily\scriptsize,language=SQL,label= ,caption= ,numbers=none}
\begin{lstlisting}
explain select *
from basic
where name='林师傅' and avg_price=12;
\end{lstlisting}

\begin{verbatim}
+----+-------------+-------+------+---------------+------+---------+------+------+-------------+
| id | select_type | table | type | possible_keys | key  | key_len | ref  | rows | Extra       |
+----+-------------+-------+------+---------------+------+---------+------+------+-------------+
| 1  | SIMPLE      | basic | ALL  | NULL          | NULL | NULL    | NULL | 963  | Using where |
+----+-------------+-------+------+---------------+------+---------+------+------+-------------+
1 rows in set (0.05 sec)
\end{verbatim}

可见 rows 为 963
\item 添加索引

\lstset{breaklines,keywordstyle=\color{black}\bfseries,basicstyle=\ttfamily\scriptsize,language=SQL,label= ,caption= ,numbers=none}
\begin{lstlisting}
create index index_name on basic(avg_price,name);
\end{lstlisting}
\item 优化后

\begin{verbatim}
+----+-------------+-------+------+---------------+------------+---------+-------------+------+-----------------------+
| id | select_type | table | type | possible_keys | key        | key_len | ref         | rows | Extra                 |
+----+-------------+-------+------+---------------+------------+---------+-------------+------+-----------------------+
| 1  | SIMPLE      | basic | ref  | index_name    | index_name | 152     | const,const | 1    | Using index condition |
+----+-------------+-------+------+---------------+------------+---------+-------------+------+-----------------------+
1 rows in set (0.04 sec)
\end{verbatim}

可见 rows 已经变为 1 ,是优化前的 0.1\%
\end{enumerate}

\item 查找特定店名但人均消费在一定范围的店铺名称
\begin{enumerate}
\item 原查询

\lstset{breaklines,keywordstyle=\color{black}\bfseries,basicstyle=\ttfamily\scriptsize,language=SQL,label= ,caption= ,numbers=none}
\begin{lstlisting}
explain select name
from basic
where name='林师傅' and avg_price<50;
\end{lstlisting}

\begin{verbatim}
+----+-------------+-------+------+---------------+------+---------+------+------+-------------+
| id | select_type | table | type | possible_keys | key  | key_len | ref  | rows | Extra       |
+----+-------------+-------+------+---------------+------+---------+------+------+-------------+
| 1  | SIMPLE      | basic | ALL  | NULL          | NULL | NULL    | NULL | 1000 | Using where |
+----+-------------+-------+------+---------------+------+---------+------+------+-------------+
1 rows in set (0.04 sec)
\end{verbatim}

可见 rows 为 1000

\item 添加索引

\lstset{breaklines,keywordstyle=\color{black}\bfseries,basicstyle=\ttfamily\scriptsize,language=SQL,label= ,caption= ,numbers=none}
\begin{lstlisting}
create index index_name_price on basic(avg_price,name);
\end{lstlisting}

\item 优化后

\begin{verbatim}
+----+-------------+-------+-------+------------------+------------------+---------+------+------+--------------------------+
| id | select_type | table | type  | possible_keys    | key              | key_len | ref  | rows | Extra                    |
+----+-------------+-------+-------+------------------+------------------+---------+------+------+--------------------------+
| 1  | SIMPLE      | basic | range | index_name_price | index_name_price | 2       | NULL | 617  | Using where; Using index |
+----+-------------+-------+-------+------------------+------------------+---------+------+------+--------------------------+
\end{verbatim}

可见 rows 已经变为 617 ,是优化前的 61.7\%
\end{enumerate}
\end{enumerate}
\item 用”in”进行条件查询
\label{sec-3-2-1-7}
\begin{enumerate}
\item 用几个 small\_cate\_id 来查询 shop\_id
\begin{enumerate}
\item 原查询

\lstset{breaklines,keywordstyle=\color{black}\bfseries,basicstyle=\ttfamily\scriptsize,language=SQL,label= ,caption= ,numbers=none}
\begin{lstlisting}
explain select shop_id
from shop_id_small_cate_id
where small_cate_id in ('g101', 'g103', 'g105', 'g107');
\end{lstlisting}

\begin{verbatim}
+----+-------------+-----------------------+-------+---------------+---------------+---------+------+------+--------------------------+
| id | select_type | table                 | type  | possible_keys | key           | key_len | ref  | rows | Extra                    |
+----+-------------+-----------------------+-------+---------------+---------------+---------+------+------+--------------------------+
|  1 | SIMPLE      | shop_id_small_cate_id | range | small_cate_id | small_cate_id | 18      | NULL |   76 | Using where; Using index |
+----+-------------+-----------------------+-------+---------------+---------------+---------+------+------+--------------------------+
1 row in set (0.01 sec)
\end{verbatim}

可见 rows 为 76

\item 无需优化

因为 small\_cate\_id 已经是 small\_cate\_id\_small\_cate 表的主键,已有索引,故
无需优化。
\end{enumerate}

\item 查找符合某几个店名的店铺
\begin{enumerate}
\item 原查询

\lstset{breaklines,keywordstyle=\color{black}\bfseries,basicstyle=\ttfamily\scriptsize,language=SQL,label= ,caption= ,numbers=none}
\begin{lstlisting}
explain select *
from basic
where name in ('真好味','阿姨布丁','青春学堂');
\end{lstlisting}

\begin{verbatim}
+----+-------------+-------+------+---------------+------+---------+------+------+-------------+
| id | select_type | table | type | possible_keys | key  | key_len | ref  | rows | Extra       |
+----+-------------+-------+------+---------------+------+---------+------+------+-------------+
| 1  | SIMPLE      | basic | ALL  | NULL          | NULL | NULL    | NULL | 963  | Using where |
+----+-------------+-------+------+---------------+------+---------+------+------+-------------+
1 rows in set (0.05 sec)
\end{verbatim}

可见 rows 为 963
\item 添加索引

\lstset{breaklines,keywordstyle=\color{black}\bfseries,basicstyle=\ttfamily\scriptsize,language=SQL,label= ,caption= ,numbers=none}
\begin{lstlisting}
create index index_name on basic(name);
\end{lstlisting}

\item 优化后

\begin{verbatim}
+----+-------------+-------+-------+---------------+------------+---------+------+------+-----------------------+
| id | select_type | table | type  | possible_keys | key        | key_len | ref  | rows | Extra                 |
+----+-------------+-------+-------+---------------+------------+---------+------+------+-----------------------+
| 1  | SIMPLE      | basic | range | index_name    | index_name | 150     | NULL | 3    | Using index condition |
+----+-------------+-------+-------+---------------+------------+---------+------+------+-----------------------+
1 rows in set (0.04 sec)
\end{verbatim}

可见 rows 已经变为 3 ,是优化前的 0.3\%
\end{enumerate}
\end{enumerate}
\item 一个表中 group by、order by、having 联合条件查询
\label{sec-3-2-1-8}
\begin{enumerate}
\item 按平均价格来查询
\begin{enumerate}
\item 原查询

\lstset{breaklines,keywordstyle=\color{black}\bfseries,basicstyle=\ttfamily\scriptsize,language=SQL,label= ,caption= ,numbers=none}
\begin{lstlisting}
explain select * from basic
where avg_price < 20
order by avg_price;
\end{lstlisting}

\begin{verbatim}
+----+-------------+-------+------+---------------+------+---------+------+------+-----------------------------+
| id | select_type | table | type | possible_keys | key  | key_len | ref  | rows | Extra                       |
+----+-------------+-------+------+---------------+------+---------+------+------+-----------------------------+
|  1 | SIMPLE      | basic | ALL  | NULL          | NULL | NULL    | NULL |  963 | Using where; Using filesort |
+----+-------------+-------+------+---------------+------+---------+------+------+-----------------------------+
1 row in set (0.00 sec)
\end{verbatim}

可见 rows 为 963

\item 利用索引进行优化

\lstset{breaklines,keywordstyle=\color{black}\bfseries,basicstyle=\ttfamily\scriptsize,language=SQL,label= ,caption= ,numbers=none}
\begin{lstlisting}
create index index_price on basic(avg_price);
\end{lstlisting}

\item 优化后

\lstset{breaklines,keywordstyle=\color{black}\bfseries,basicstyle=\ttfamily\scriptsize,language=SQL,label= ,caption= ,numbers=none}
\begin{lstlisting}
explain select * from basic
where avg_price < 20
order by avg_price;
\end{lstlisting}

\begin{verbatim}
+----+-------------+-------+-------+---------------+-------------+---------+------+------+-----------------------+
| id | select_type | table | type  | possible_keys | key         | key_len | ref  | rows | Extra                 |
+----+-------------+-------+-------+---------------+-------------+---------+------+------+-----------------------+
|  1 | SIMPLE      | basic | range | index_price   | index_price | 2       | NULL |  235 | Using index condition |
+----+-------------+-------+-------+---------------+-------------+---------+------+------+-----------------------+
1 row in set (0.04 sec)
\end{verbatim}
\end{enumerate}

\item remark 表分组排序筛选查找
\begin{enumerate}
\item 原查询

将 remark 按环境评分 environment\_rating 分组,筛选出食品评分 product\_rating
> 7 的餐馆, 并且按平均评价数量 all\_remarks 排序

\lstset{breaklines,keywordstyle=\color{black}\bfseries,basicstyle=\ttfamily\scriptsize,language=SQL,label= ,caption= ,numbers=none}
\begin{lstlisting}
explain select shop_id,environment_rating,product_rating,all_remarks
from remark
group by environment_rating
having product_rating>7
order by all_remarks;
\end{lstlisting}

\begin{verbatim}
+----+-------------+--------+------+---------------+------+---------+------+------+---------------------------------+
| id | select_type | table  | type | possible_keys | key  | key_len | ref  | rows | Extra                           |
+----+-------------+--------+------+---------------+------+---------+------+------+---------------------------------+
| 1  | SIMPLE      | remark | ALL  | NULL          | NULL | NULL    | NULL | 1000 | Using temporary; Using filesort |
+----+-------------+--------+------+---------------+------+---------+------+------+---------------------------------+
1 rows in set (0.03 sec)
\end{verbatim}

可见 rows 为 1000
\item 添加索引

\lstset{breaklines,keywordstyle=\color{black}\bfseries,basicstyle=\ttfamily\scriptsize,language=SQL,label= ,caption= ,numbers=none}
\begin{lstlisting}
create index index_environment_rating,product_rating,all_remarks on remark(environment_rating,product_rating,all_remarks);
\end{lstlisting}
\item 优化后

\begin{verbatim}
+----+-------------+--------+-------+---------------+---------+---------+------+------+-----------------------------------------------------------+
| id | select_type | table  | type  | possible_keys | key     | key_len | ref  | rows | Extra                                                     |
+----+-------------+--------+-------+---------------+---------+---------+------+------+-----------------------------------------------------------+
| 1  | SIMPLE      | remark | range | index_1       | index_1 | 5       | NULL | 84   | Using index for group-by; Using temporary; Using filesort |
+----+-------------+--------+-------+---------------+---------+---------+------+------+-----------------------------------------------------------+
1 rows in set (0.06 sec)
\end{verbatim}

可见 rows 已经变为 84 ,是优化前的 0.84\%
\end{enumerate}
\end{enumerate}
\end{itemize}
\subsubsection{复合查询}
\label{sec-3-2-2}
\begin{itemize}
\item 多表联合查询
\label{sec-3-2-2-1}
\begin{enumerate}
\item 查询某 city 的所有 shop\_id
\begin{enumerate}
\item 原查询

\lstset{breaklines,keywordstyle=\color{black}\bfseries,basicstyle=\ttfamily\scriptsize,language=SQL,label= ,caption= ,numbers=none}
\begin{lstlisting}
explain select shop_id
from shop_id_city_id as SC, city_id_city as C
where SC.city_id = C.city_id and C.city = '上海';
\end{lstlisting}

\begin{verbatim}
+----+-------------+-------+------+---------------+---------+---------+----------------+------+-------------+
| id | select_type | table | type | possible_keys | key     | key_len | ref            | rows | Extra       |
+----+-------------+-------+------+---------------+---------+---------+----------------+------+-------------+
|  1 | SIMPLE      | C     | ALL  | PRIMARY       | NULL    | NULL    | NULL           |   59 | Using where |
|  1 | SIMPLE      | SC    | ref  | city_id       | city_id | 2       | test.C.city_id |    1 | Using index |
+----+-------------+-------+------+---------------+---------+---------+----------------+------+-------------+
2 rows in set (0.01 sec)
\end{verbatim}

可见 rows 为 59 * 1 ,因 city\_id 为主键,故在 SC 表上无需优化

\item 添加索引

\lstset{breaklines,keywordstyle=\color{black}\bfseries,basicstyle=\ttfamily\scriptsize,language=SQL,label= ,caption= ,numbers=none}
\begin{lstlisting}
create index index_city on city_id_city(city);
\end{lstlisting}

\item 优化后

\lstset{breaklines,keywordstyle=\color{black}\bfseries,basicstyle=\ttfamily\scriptsize,language=SQL,label= ,caption= ,numbers=none}
\begin{lstlisting}
explain select shop_id
from shop_id_city_id as SC, city_id_city as C
where SC.city_id = C.city_id and C.city = '上海';
\end{lstlisting}

\begin{verbatim}
+----+-------------+-------+------+--------------------+------------+---------+----------------+------+--------------------------+
| id | select_type | table | type | possible_keys      | key        | key_len | ref            | rows | Extra                    |
+----+-------------+-------+------+--------------------+------------+---------+----------------+------+--------------------------+
|  1 | SIMPLE      | C     | ref  | PRIMARY,index_city | index_city | 45      | const          |    1 | Using where; Using index |
|  1 | SIMPLE      | SC    | ref  | city_id            | city_id    | 2       | test.C.city_id |    1 | Using index              |
+----+-------------+-------+------+--------------------+------------+---------+----------------+------+--------------------------+
2 rows in set (0.01 sec)
\end{verbatim}

可见 rows 变为 1 * 1 ,效率显著提高
\end{enumerate}

\item 查询某小类下的所有商铺名
\begin{enumerate}
\item 原查询

\lstset{breaklines,keywordstyle=\color{black}\bfseries,basicstyle=\ttfamily\scriptsize,language=SQL,label= ,caption= ,numbers=none}
\begin{lstlisting}
explain select B.shop_id, B.name
from basic as B, shop_id_small_cate_id as SS
where B.shop_id = SS.shop_id and SS.small_cate_id = 'g101';
\end{lstlisting}

\begin{verbatim}
+----+-------------+-------+--------+-----------------------+---------------+---------+-----------------+------+--------------------------+
| id | select_type | table | type   | possible_keys         | key           | key_len | ref             | rows | Extra                    |
+----+-------------+-------+--------+-----------------------+---------------+---------+-----------------+------+--------------------------+
|  1 | SIMPLE      | SS    | ref    | PRIMARY,small_cate_id | small_cate_id | 18      | const           |   32 | Using where; Using index |
|  1 | SIMPLE      | B     | eq_ref | PRIMARY               | PRIMARY       | 4       | test.SS.shop_id |    1 | NULL                     |
+----+-------------+-------+--------+-----------------------+---------------+---------+-----------------+------+--------------------------+
2 rows in set (0.01 sec)
\end{verbatim}

rows 为 32 * 1

\item 无需优化

因为 small\_cate\_id 和 shop\_id 都是主键,已有索引,故无需优化
\end{enumerate}

\item 寻找平均价格低于 20 元且食物评分高于 8 的店铺
\begin{enumerate}
\item 原查询

\lstset{breaklines,keywordstyle=\color{black}\bfseries,basicstyle=\ttfamily\scriptsize,language=SQL,label= ,caption= ,numbers=none}
\begin{lstlisting}
explain select basic.name,basic.avg_price,remark.product_rating
from basic,remark
where basic.shop_id=remark.shop_id and basic.avg_price<20 and remark.product_rating>8;
\end{lstlisting}

\begin{verbatim}
+----+-------------+--------+--------+---------------+---------+---------+--------------------+------+-------------+
| id | select_type | table  | type   | possible_keys | key     | key_len | ref                | rows | Extra       |
+----+-------------+--------+--------+---------------+---------+---------+--------------------+------+-------------+
| 1  | SIMPLE      | basic  | ALL    | PRIMARY       | NULL    | NULL    | NULL               | 1000 | Using where |
| 1  | SIMPLE      | remark | eq_ref | PRIMARY       | PRIMARY | 4       | test.basic.shop_id | 1    | Using where |
+----+-------------+--------+--------+---------------+---------+---------+--------------------+------+-------------+
2 rows in set (0.06 sec)
\end{verbatim}

可见 rows 分别为 1000, 1
\item 添加索引

\lstset{breaklines,keywordstyle=\color{black}\bfseries,basicstyle=\ttfamily\scriptsize,language=SQL,label= ,caption= ,numbers=none}
\begin{lstlisting}
create index index_avg_price on basic(avg_price);
\end{lstlisting}
\item 优化后

\begin{verbatim}
+----+-------------+--------+--------+-------------------------+-----------------+---------+--------------------+------+-----------------------+
| id | select_type | table  | type   | possible_keys           | key             | key_len | ref                | rows | Extra                 |
+----+-------------+--------+--------+-------------------------+-----------------+---------+--------------------+------+-----------------------+
| 1  | SIMPLE      | basic  | range  | PRIMARY,index_avg_price | index_avg_price | 2       | NULL               | 235  | Using index condition |
| 1  | SIMPLE      | remark | eq_ref | PRIMARY                 | PRIMARY         | 4       | test.basic.shop_id | 1    | Using where           |
+----+-------------+--------+--------+-------------------------+-----------------+---------+--------------------+------+-----------------------+
2 rows in set (0.05 sec)
\end{verbatim}

可见 rows 已经变为 235, 1 ,是优化前的 23.5\%
\end{enumerate}
\end{enumerate}
\item join 查询
\label{sec-3-2-2-2}
\begin{enumerate}
\item 利用拼音查询城市名
\begin{enumerate}
\item 原查询

\lstset{breaklines,keywordstyle=\color{black}\bfseries,basicstyle=\ttfamily\scriptsize,language=SQL,label= ,caption= ,numbers=none}
\begin{lstlisting}
explain select city
from city_id_city as C
inner join city_id_city_pinyin as CP
where C.city_id = CP.city_id and CP.city_pinyin = 'shanghai';
\end{lstlisting}

\begin{verbatim}
+----+-------------+-------+--------+---------------+---------+---------+----------------+------+-------------+
| id | select_type | table | type   | possible_keys | key     | key_len | ref            | rows | Extra       |
+----+-------------+-------+--------+---------------+---------+---------+----------------+------+-------------+
|  1 | SIMPLE      | C     | ALL    | PRIMARY       | NULL    | NULL    | NULL           |   59 | NULL        |
|  1 | SIMPLE      | CP    | eq_ref | PRIMARY       | PRIMARY | 2       | test.C.city_id |    1 | Using where |
+----+-------------+-------+--------+---------------+---------+---------+----------------+------+-------------+
2 rows in set (0.01 sec)
\end{verbatim}

可见 rows 为 59 * 1 ,因 city\_id 为主键,故在 C 表上无需优化

\item 添加索引

\lstset{breaklines,keywordstyle=\color{black}\bfseries,basicstyle=\ttfamily\scriptsize,language=SQL,label= ,caption= ,numbers=none}
\begin{lstlisting}
create index index_city_pinyin on city_id_city_pinyin(city_pinyin);
\end{lstlisting}

\item 优化后

\lstset{breaklines,keywordstyle=\color{black}\bfseries,basicstyle=\ttfamily\scriptsize,language=SQL,label= ,caption= ,numbers=none}
\begin{lstlisting}
explain select city
from city_id_city as C
inner join city_id_city_pinyin as CP
where C.city_id = CP.city_id and CP.city_pinyin = 'shanghai';
\end{lstlisting}

\begin{verbatim}
+----+-------------+-------+--------+---------------------------+-------------------+---------+-----------------+------+--------------------------+
| id | select_type | table | type   | possible_keys             | key               | key_len | ref             | rows | Extra                    |
+----+-------------+-------+--------+---------------------------+-------------------+---------+-----------------+------+--------------------------+
|  1 | SIMPLE      | CP    | ref    | PRIMARY,index_city_pinyin | index_city_pinyin | 45      | const           |    1 | Using where; Using index |
|  1 | SIMPLE      | C     | eq_ref | PRIMARY                   | PRIMARY           | 2       | test.CP.city_id |    1 | NULL                     |
+----+-------------+-------+--------+---------------------------+-------------------+---------+-----------------+------+--------------------------+
2 rows in set (0.01 sec)
\end{verbatim}

rows 变为 1 * 1 ,效率显著提高
\end{enumerate}

\item 获取所有商铺的图片
\begin{enumerate}
\item 原查询

\lstset{breaklines,keywordstyle=\color{black}\bfseries,basicstyle=\ttfamily\scriptsize,language=SQL,label= ,caption= ,numbers=none}
\begin{lstlisting}
explain select B.shop_id, D.photos
from basic as B
inner join dazhong as D
where B.shop_id = D.shop_id;
\end{lstlisting}

\begin{verbatim}
+----+-------------+-------+--------+---------------+---------+---------+----------------+------+-------------+
| id | select_type | table | type   | possible_keys | key     | key_len | ref            | rows | Extra       |
+----+-------------+-------+--------+---------------+---------+---------+----------------+------+-------------+
|  1 | SIMPLE      | D     | ALL    | PRIMARY       | NULL    | NULL    | NULL           |  939 | NULL        |
|  1 | SIMPLE      | B     | eq_ref | PRIMARY       | PRIMARY | 4       | test.D.shop_id |    1 | Using index |
+----+-------------+-------+--------+---------------+---------+---------+----------------+------+-------------+
2 rows in set (0.43 sec)
\end{verbatim}

可见 rows 为 939 * 1

\item 无法优化

因为每个商铺的图片地址均不相同,故读取所有地址需要遍历全表,无法优化
\end{enumerate}
\end{enumerate}
\item 存在量词(exists)查询
\label{sec-3-2-2-3}
\begin{enumerate}
\item 查询产品评价高于 9 的 shop\_id, name
\begin{enumerate}
\item 原查询

\lstset{breaklines,keywordstyle=\color{black}\bfseries,basicstyle=\ttfamily\scriptsize,language=SQL,label= ,caption= ,numbers=none}
\begin{lstlisting}
explain select B.shop_id, B.name
from basic as B
where exists
(select *
from remark as R
where R.shop_id = B.shop_id and R.product_rating > 9);
\end{lstlisting}

\begin{verbatim}
+----+--------------------+-------+--------+---------------+---------+---------+----------------+------+-------------+
| id | select_type        | table | type   | possible_keys | key     | key_len | ref            | rows | Extra       |
+----+--------------------+-------+--------+---------------+---------+---------+----------------+------+-------------+
|  1 | PRIMARY            | B     | ALL    | NULL          | NULL    | NULL    | NULL           |  963 | Using where |
|  2 | DEPENDENT SUBQUERY | R     | eq_ref | PRIMARY       | PRIMARY | 4       | test.B.shop_id |    1 | Using where |
+----+--------------------+-------+--------+---------------+---------+---------+----------------+------+-------------+
2 rows in set (0.01 sec)
\end{verbatim}

可见 rows 为 963 * 1

\item 添加索引

因 shop\_id 已有索引,故不再添加

\lstset{breaklines,keywordstyle=\color{black}\bfseries,basicstyle=\ttfamily\scriptsize,language=SQL,label= ,caption= ,numbers=none}
\begin{lstlisting}
create index index_product_rating on remark(product_rating);
\end{lstlisting}

但添加索引后结果不变,考虑是 exists 的问题

\item 改为 join 查询

\lstset{breaklines,keywordstyle=\color{black}\bfseries,basicstyle=\ttfamily\scriptsize,language=SQL,label= ,caption= ,numbers=none}
\begin{lstlisting}
explain select B.shop_id, B.name
from basic as B, remark as R
where R.shop_id = B.shop_id and R.product_rating > 9;
\end{lstlisting}

\begin{verbatim}
+----+-------------+-------+--------+------------------------------+----------------------+---------+----------------+------+--------------------------+
| id | select_type | table | type   | possible_keys                | key                  | key_len | ref            | rows | Extra                    |
+----+-------------+-------+--------+------------------------------+----------------------+---------+----------------+------+--------------------------+
|  1 | SIMPLE      | R     | range  | PRIMARY,index_product_rating | index_product_rating | 5       | NULL           |    5 | Using where; Using index |
|  1 | SIMPLE      | B     | eq_ref | PRIMARY                      | PRIMARY              | 4       | test.R.shop_id |    1 | NULL                     |
+----+-------------+-------+--------+------------------------------+----------------------+---------+----------------+------+--------------------------+
2 rows in set (0.05 sec)
\end{verbatim}

经过添加索引和改为 join 查询后, rows 变为 5 * 1 ,效率显著提高
\end{enumerate}
\end{enumerate}
\item 嵌套子查询(select \ldots{} from (select \ldots{}))
\label{sec-3-2-2-4}
\begin{enumerate}
\item 查找 stars 大于 4.0 的 shop\_id
\begin{enumerate}
\item 原查询

\lstset{breaklines,keywordstyle=\color{black}\bfseries,basicstyle=\ttfamily\scriptsize,language=SQL,label= ,caption= ,numbers=none}
\begin{lstlisting}
select shop_id
from (select * from dazhong where stars > 4.0) as D;
\end{lstlisting}

\begin{verbatim}
+----+-------------+------------+------+---------------+------+---------+------+------+-------------+
| id | select_type | table      | type | possible_keys | key  | key_len | ref  | rows | Extra       |
+----+-------------+------------+------+---------------+------+---------+------+------+-------------+
|  1 | PRIMARY     | <derived2> | ALL  | NULL          | NULL | NULL    | NULL |  958 | NULL        |
|  2 | DERIVED     | dazhong    | ALL  | NULL          | NULL | NULL    | NULL |  958 | Using where |
+----+-------------+------------+------+---------------+------+---------+------+------+-------------+
2 rows in set (0.03 sec)
\end{verbatim}

可见 rows 为 958 * 958 ,效率极低

\item 添加索引

\lstset{breaklines,keywordstyle=\color{black}\bfseries,basicstyle=\ttfamily\scriptsize,language=SQL,label= ,caption= ,numbers=none}
\begin{lstlisting}
create index index_stars on dazhong(stars);
\end{lstlisting}

添加索引后

\lstset{breaklines,keywordstyle=\color{black}\bfseries,basicstyle=\ttfamily\scriptsize,language=SQL,label= ,caption= ,numbers=none}
\begin{lstlisting}
explain select shop_id
from (select * from dazhong where stars > 4.0) as D;
\end{lstlisting}

\begin{verbatim}
+----+-------------+------------+-------+---------------+-------------+---------+------+------+-----------------------+
| id | select_type | table      | type  | possible_keys | key         | key_len | ref  | rows | Extra                 |
+----+-------------+------------+-------+---------------+-------------+---------+------+------+-----------------------+
|  1 | PRIMARY     | <derived2> | ALL   | NULL          | NULL        | NULL    | NULL |   86 | NULL                  |
|  2 | DERIVED     | dazhong    | range | index_stars   | index_stars | 5       | NULL |   86 | Using index condition |
+----+-------------+------------+-------+---------------+-------------+---------+------+------+-----------------------+
2 rows in set (0.02 sec)
\end{verbatim}

rows 降为 86 * 86 ,效率在一定程度上提高

\item 改为非嵌套查询

\lstset{breaklines,keywordstyle=\color{black}\bfseries,basicstyle=\ttfamily\scriptsize,language=SQL,label= ,caption= ,numbers=none}
\begin{lstlisting}
explain select shop_id
from dazhong where stars > 4.0;
\end{lstlisting}

\begin{verbatim}
+----+-------------+---------+-------+---------------+-------------+---------+------+------+--------------------------+
| id | select_type | table   | type  | possible_keys | key         | key_len | ref  | rows | Extra                    |
+----+-------------+---------+-------+---------------+-------------+---------+------+------+--------------------------+
|  1 | SIMPLE      | dazhong | range | index_stars   | index_stars | 5       | NULL |   86 | Using where; Using index |
+----+-------------+---------+-------+---------------+-------------+---------+------+------+--------------------------+
1 row in set (0.01 sec)
\end{verbatim}

经过添加索引、改为非嵌套查询优化后, rows 降为 86 ,效率显著提高
\end{enumerate}

\item 查找所有位于杨浦区的店铺 id 和名称与平均价格

\begin{enumerate}
\item 原查询

\lstset{breaklines,keywordstyle=\color{black}\bfseries,basicstyle=\ttfamily\scriptsize,language=SQL,label= ,caption= ,numbers=none}
\begin{lstlisting}
explain select shop_id,name,avg_price
from basic
where shop_id in(select shop_id
from shop_id_area
where area='杨浦区');
\end{lstlisting}

\begin{verbatim}
+----+-------------+--------------+--------+---------------+---------+---------+---------------------------+------+-------------+
| id | select_type | table        | type   | possible_keys | key     | key_len | ref                       | rows | Extra       |
+----+-------------+--------------+--------+---------------+---------+---------+---------------------------+------+-------------+
| 1  | SIMPLE      | shop_id_area | ALL    | PRIMARY       | NULL    | NULL    | NULL                      | 1000 | Using where |
| 1  | SIMPLE      | basic        | eq_ref | PRIMARY       | PRIMARY | 4       | test.shop_id_area.shop_id | 1    | NULL        |
+----+-------------+--------------+--------+---------------+---------+---------+---------------------------+------+-------------+
2 rows in set (0.04 sec)
\end{verbatim}

可见 rows 分别为 1000,1

\item 添加索引

对 shop\_id\_area 的 area 进行索引

\lstset{breaklines,keywordstyle=\color{black}\bfseries,basicstyle=\ttfamily\scriptsize,language=SQL,label= ,caption= ,numbers=none}
\begin{lstlisting}
create index index_shop_id_area on shop_id_area(area);
\end{lstlisting}

\item 优化后

\begin{verbatim}
+----+-------------+--------------+--------+----------------------------+--------------------+---------+---------------------------+------+--------------------------+
| id | select_type | table        | type   | possible_keys              | key                | key_len | ref                       | rows | Extra                    |
+----+-------------+--------------+--------+----------------------------+--------------------+---------+---------------------------+------+--------------------------+
| 1  | SIMPLE      | shop_id_area | ref    | PRIMARY,index_shop_id_area | index_shop_id_area | 120     | const                     | 19   | Using where; Using index |
| 1  | SIMPLE      | basic        | eq_ref | PRIMARY                    | PRIMARY            | 4       | test.shop_id_area.shop_id | 1    | NULL                     |
+----+-------------+--------------+--------+----------------------------+--------------------+---------+---------------------------+------+--------------------------+
2 rows in set (0.03 sec)
\end{verbatim}

可见 rows 已经变为 19, 1 ,是优化前的 1.9\%
\end{enumerate}
\end{enumerate}
\end{itemize}
\subsubsection{其他查询}
\label{sec-3-2-3}
\begin{itemize}
\item 向表中插入记录
\label{sec-3-2-3-1}
\begin{enumerate}
\item 直接向 city\_id\_city 中插入新元组
\begin{enumerate}
\item 原查询

\lstset{breaklines,keywordstyle=\color{black}\bfseries,basicstyle=\ttfamily\scriptsize,language=SQL,label= ,caption= ,numbers=none}
\begin{lstlisting}
explain insert into city_id_city(city_id, city)
values (1001, 'test');
\end{lstlisting}

\begin{verbatim}
+----+-------------+-------+------+---------------+------+---------+------+------+----------------+
| id | select_type | table | type | possible_keys | key  | key_len | ref  | rows | Extra          |
+----+-------------+-------+------+---------------+------+---------+------+------+----------------+
|  1 | SIMPLE      | NULL  | NULL | NULL          | NULL | NULL    | NULL | NULL | No tables used |
+----+-------------+-------+------+---------------+------+---------+------+------+----------------+
1 row in set (0.00 sec)
\end{verbatim}

直接插入无需优化
\end{enumerate}
\end{enumerate}
\item 删除记录
\label{sec-3-2-3-2}
\begin{enumerate}
\item 直接将 city\_id\_city 中的一项删除
\begin{enumerate}
\item 原查询

\lstset{breaklines,keywordstyle=\color{black}\bfseries,basicstyle=\ttfamily\scriptsize,language=SQL,label= ,caption= ,numbers=none}
\begin{lstlisting}
explain delete
from city_id_city
where city = 'test';
\end{lstlisting}

\begin{verbatim}
+----+-------------+--------------+------+---------------+------+---------+------+------+-------------+
| id | select_type | table        | type | possible_keys | key  | key_len | ref  | rows | Extra       |
+----+-------------+--------------+------+---------------+------+---------+------+------+-------------+
|  1 | SIMPLE      | city_id_city | ALL  | NULL          | NULL | NULL    | NULL |   60 | Using where |
+----+-------------+--------------+------+---------------+------+---------+------+------+-------------+
1 row in set (0.01 sec)
\end{verbatim}

rows 为 60

\item 添加索引

\lstset{breaklines,keywordstyle=\color{black}\bfseries,basicstyle=\ttfamily\scriptsize,language=SQL,label= ,caption= ,numbers=none}
\begin{lstlisting}
create index index_city on city_id_city(city);
\end{lstlisting}

添加索引后

\lstset{breaklines,keywordstyle=\color{black}\bfseries,basicstyle=\ttfamily\scriptsize,language=SQL,label= ,caption= ,numbers=none}
\begin{lstlisting}
explain delete
from city_id_city
where city = 'test';
\end{lstlisting}

\begin{verbatim}
+----+-------------+--------------+-------+---------------+------------+---------+-------+------+-------------+
| id | select_type | table        | type  | possible_keys | key        | key_len | ref   | rows | Extra       |
+----+-------------+--------------+-------+---------------+------------+---------+-------+------+-------------+
|  1 | SIMPLE      | city_id_city | range | index_city    | index_city | 45      | const |    1 | Using where |
+----+-------------+--------------+-------+---------------+------------+---------+-------+------+-------------+
1 row in set (0.01 sec)
\end{verbatim}

rows 变为 1 ,效率显著提高

\item 使用主键

使用主键来进行删除操作

\lstset{breaklines,keywordstyle=\color{black}\bfseries,basicstyle=\ttfamily\scriptsize,language=SQL,label= ,caption= ,numbers=none}
\begin{lstlisting}
explain delete
from city_id_city
where city_id = 1001 and city = 'test';
\end{lstlisting}

\begin{verbatim}
+----+-------------+--------------+-------+---------------+---------+---------+-------+------+-------------+
| id | select_type | table        | type  | possible_keys | key     | key_len | ref   | rows | Extra       |
+----+-------------+--------------+-------+---------------+---------+---------+-------+------+-------------+
|  1 | SIMPLE      | city_id_city | range | PRIMARY       | PRIMARY | 2       | const |    1 | Using where |
+----+-------------+--------------+-------+---------------+---------+---------+-------+------+-------------+
1 row in set (0.03 sec)
\end{verbatim}

rows 也为 1 ,效率显著提高
\end{enumerate}
\end{enumerate}
\item 聚集函数
\label{sec-3-2-3-3}
\begin{enumerate}
\item 统计某城市中有多少商铺
\begin{enumerate}
\item 原查询

\lstset{breaklines,keywordstyle=\color{black}\bfseries,basicstyle=\ttfamily\scriptsize,language=SQL,label= ,caption= ,numbers=none}
\begin{lstlisting}
explain select count(city_id)
from shop_id_city_id
where city_id = 1;
\end{lstlisting}

\begin{verbatim}
+----+-------------+-----------------+------+---------------+---------+---------+-------+------+-------------+
| id | select_type | table           | type | possible_keys | key     | key_len | ref   | rows | Extra       |
+----+-------------+-----------------+------+---------------+---------+---------+-------+------+-------------+
|  1 | SIMPLE      | shop_id_city_id | ref  | city_id       | city_id | 2       | const |  257 | Using index |
+----+-------------+-----------------+------+---------------+---------+---------+-------+------+-------------+
1 row in set (0.00 sec)
\end{verbatim}

rows 为 257 ,但已经是利用了主键索引了,且 count 函数是需要遍历所有 257 个
结果的,故已经无法再优化了

\item 实际应用

在实际应用 count 语句时,使用 \texttt{count(*)} 将比 count(\emph{具体键值}) 快。因为
\texttt{count(*)} 利用的是主键索引\footnote{\url{http://blog.itpub.net/22664653/viewspace-774679/}}。
\end{enumerate}
\end{enumerate}
\item 其他查询
\label{sec-3-2-3-4}
\begin{enumerate}
\item Like 语句
\begin{enumerate}
\item 查找所有星巴克的分店
\begin{enumerate}
\item 原查询

\lstset{breaklines,keywordstyle=\color{black}\bfseries,basicstyle=\ttfamily\scriptsize,language=SQL,label= ,caption= ,numbers=none}
\begin{lstlisting}
explain select shop_id,name
from basic
where name like '星巴克%';
\end{lstlisting}

\begin{verbatim}
+----+-------------+-------+------+---------------+------+---------+------+------+-------------+
| id | select_type | table | type | possible_keys | key  | key_len | ref  | rows | Extra       |
+----+-------------+-------+------+---------------+------+---------+------+------+-------------+
| 1  | SIMPLE      | basic | ALL  | NULL          | NULL | NULL    | NULL | 963  | Using where |
+----+-------------+-------+------+---------------+------+---------+------+------+-------------+
1 rows in set (0.03 sec)
\end{verbatim}

可见 rows 为 963

\item 添加索引

\lstset{breaklines,keywordstyle=\color{black}\bfseries,basicstyle=\ttfamily\scriptsize,language=SQL,label= ,caption= ,numbers=none}
\begin{lstlisting}
create index index_name on basic(name);
\end{lstlisting}

\item 优化后

\begin{verbatim}
+----+-------------+-------+-------+---------------+------------+---------+------+------+--------------------------+
| id | select_type | table | type  | possible_keys | key        | key_len | ref  | rows | Extra                    |
+----+-------------+-------+-------+---------------+------------+---------+------+------+--------------------------+
| 1  | SIMPLE      | basic | range | index_name    | index_name | 150     | NULL | 6    | Using where; Using index |
+----+-------------+-------+-------+---------------+------------+---------+------+------+--------------------------+
1 rows in set (0.03 sec)
\end{verbatim}

可见 rows 已经变为 6 ,是优化前的 0.6\%

\item 优化查询语句

将 Like 查询转化为范围查询效率更高 (「兌」是「克」在 utf-8 编码下的后一位)

\lstset{breaklines,keywordstyle=\color{black}\bfseries,basicstyle=\ttfamily\scriptsize,language=SQL,label= ,caption= ,numbers=none}
\begin{lstlisting}
select shop_id, name
from basic
where name>'星巴克' and name<'星巴兌';
\end{lstlisting}
\end{enumerate}
\end{enumerate}
\end{enumerate}
\end{itemize}
% Emacs 24.5.1 (Org mode 8.2.10)
\end{document}